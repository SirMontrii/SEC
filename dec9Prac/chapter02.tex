\documentclass{book}
\title{BOOK}
\begin{document}
\maketitle
\chapter*{Preface}
\section*{Purpose}
The use of probability models and statistical methods for analyzing data has become common practice in virtually all scientific disciplines. This book attempts to provide a comprehensive introduction to those models and methods most likely to be encoun- tered and used by students in their careers in engineering and the natural sciences. Although the examples and exercises have been designed with scientists and engi- neers in mind, most of the methods covered are basic to statistical analyses in many other disciplines, so that students of business and the social sciences will also profit from reading the book.
\section*{Approach}
Students in a statistics course designed to serve other majors may be initially skeptical of the value and relevance of the subject matter, but my experience is that students can be turned on to statistics by the use of good examples and exercises that blend their every- day experiences with their scientific interests. Consequently, I have worked hard to find examples of real, rather than artificial, data-data that someone thought was worth col- lecting and analyzing. Many of the methods presented, especially in the later chapters on statistical inference, are illustrated by analyzing data taken from published sources, and many of the exercises also involve working with such data. Sometimes the reader may be unfamiliar with the context of a particular problem (as indeed I often was), but I have found that students are more attracted by real problems with a somewhat strange context than by patently artificial problems in a familiar setting.
\section*{Mathematical Level}
The exposition is relatively modest in terms of mathematical development. Substantial use of the calculus is made only in Chapter 4 and parts of Chapters 5 and 6. In particu- lar, with the exception of an occasional remark or aside, calculus appears in the inference part of the book only in the second section of Chapter 6. Matrix algebra is not used at all. Thus almost all the exposition should be accessible to those whose mathematical background includes one semester or two quarters of differential and integral calculus. 
\section*{Content}
Chapter 1 begins with some basic concepts and terminology population, sample, descriptive and inferential statistics, enumerative versus analytic studies, and so on- and continues with a survey of important graphical and numerical descriptive methods. A rather traditional development of probability is given in Chapter 2, followed by prob- ability distributions of discrete and continuous random variables in Chapters 3 and 4, respectively. Joint distributions and their properties are discussed in the first part of Chapter 5. The latter part of this chapter introduces statistics and their sampling distri- butions, which form the bridge between probability and inference. The next three chapters cover point estimation, statistical intervals, and hypothesis testing based on a single sample. Methods of inference involving two independent samples and paired data are presented in Chapter 9. The analysis of variance is the subject of Chapters 10 and 11 (single-factor and multifactor, respectively). Regression makes its initial appearance in Chapter 12 (the simple linear regression model and correlation) and
xiii

\chapter{Overview and Descriptive Statistics}
\section*{Introduction}
Statistical concepts and methods are not only useful but indeed often indispensable in  understanding the world around us. They provide ways of gaining new insights into the behaviour of many phenomenon that you will encounter in your chosen field of specialization in engineering or science.
\section {Populations, Samples and Processes}
Students in a statistics course designed to serve other majors may be initially skeptical of the value and relevance of the subject matter, but my experience is that students can be turned on to statistics by the use of good examples and exercises that blend their every- day experiences with their scientific interests. Consequently, I have worked hard to find examples of real, rather than artificial, data-data that someone thought was worth col- lecting and analyzing. Many of the methods presented, especially in the later chapters on statistical inference, are illustrated by analyzing data taken from published sources, and many of the exercises also involve working with such data. 

Chapter 1 begins with some basic concepts and terminology population, sample, descriptive and inferential statistics, enumerative versus analytic studies, and so on- and continues with a survey of important graphical and numerical descriptive methods. A rather traditional development of probability is given in Chapter 2, followed by prob- ability distributions of discrete and continuous random variables in Chapters 3 and 4, respectively. Joint distributions and their properties are discussed in the first part of Chapter 5. The latter part of this chapter introduces statistics and their sampling distri- butions, which form the bridge between probability and inference. The next three chapters cover point estimation, statistical intervals, and hypothesis testing based on a single sample. Methods of inference involving two independent samples and paired data are presented in Chapter 9. The analysis of variance is the subject of Chapters 10 and 11 (single-factor and multifactor, respectively). Regression makes its 
\section{Pictorial and Tabular Methods in Descriptive Statistics}
The exposition is relatively modest in terms of mathematical development. Substantial use of the calculus is made only in Chapter 4 and parts of Chapters 5 and 6. In particu- lar, with the exception of an occasional remark or aside, calculus appears in the inference part of the book only in the second section of Chapter 6. Matrix algebra is not used at all. Thus almost all the exposition should be accessible to those whose mathematical background includes one semester or two quarters of differential and integral calculus. 
\subsection{Notation}
The use of probability models and statistical methods for analyzing data has become common practice in virtually all scientific disciplines. This book attempts to provide a comprehensive introduction to those models and methods most likely to be encountered and used by students in their careers in engineering and the natural sciences. Although the examples and exercises have been designed with scientists and engi- neers in mind, most of the methods covered are basic to statistical analyses in many other disciplines, so that students of business and the social sciences will also profit from reading the book.
\end{document}
