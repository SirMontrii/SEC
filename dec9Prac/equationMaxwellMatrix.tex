\documentclass{article}
\usepackage{amsmath}

\title{}
\author{}
\date{17 April 2023}

\begin{document}
\maketitle
\section{Equation}
\textbf{Hello World!} Today I am learning \LaTeX. \LaTeX{} is a great program for writing math such as $a^2 + b^2 = c^2$. I can also give equations their own space.
\[
\gamma^2 + \theta^2 = \omega^2
\]
\section{Maxwell's Equations}
"Maxwell's equations" are named for James Clark Maxwell and are as follows:
By Gauss's Law
\begin{equation}
    \vec{\nabla} \cdot \vec{E} = \dfrac{\rho}{\epsilon_0}
\end{equation}
By Gauss's Law for Magnetism
\begin{equation}
    \vec{\nabla} \cdot \vec{B} = 0
\end{equation}
By Faraday's Law for Magnetism
\begin{equation}
    \vec{\nabla} \times \vec{E} = -\frac{\partial \vec{B}}{\partial t}
\end{equation}
By Ampere's Circuital Law
\begin{equation}
    \vec{\nabla} \times \vec{B} = \mu_0 \left(\epsilon_0 \frac{\partial \vec{E}}{\partial t} + \vec{J}\right)
\end{equation}

\section{Matrix Equation}
\begin{equation*}
\begin{pmatrix}
    a_{11} & a_{12} & \dots & a_{1n} \\
    a_{21} & a_{22} & \dots & a_{2n} \\
    \vdots & \vdots & \ddots & \vdots \\
    a_{n1} & a_{n2} & \dots & a_{nn}
\end{pmatrix}
\begin{bmatrix}
    v_{1} \\
    v_{2} \\
    \vdots \\
    v_{n}
\end{bmatrix}
=
\begin{bmatrix}
    w_{1} \\
    w_{2} \\
    \vdots \\
    w_{n}
\end{bmatrix}
\end{equation*}

\end{document}
